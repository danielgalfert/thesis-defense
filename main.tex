% VIGTIGT! For at dokumentet kan compile, skal man benytte XeLaTeX eller LuaLaTeX som compiler. Dette kan gøres i Overleaf ved at gå til Menu -> Settings -> Compiler -> Vælg XeLaTeX/LuaLaTeX
\documentclass[t,24pt]{beamer}

\usepackage{KUstyle}

\toplinje{Master Thesis Defense} %Teksten på toplinjen. Hvis ingen tekst ønskes, skal kommandoen fjernes

\begin{document}

% Det første slide. Her kan man f.eks. ændre hovedtitel, undertitlen, oplægsholder, KU-enhed og dato
{
\setbeamertemplate{background}{\includegraphics[width=\paperwidth,height=\paperheight]{KU/forside.pdf}}
\begin{frame}
    \begin{textblock*}{\textwidth}(0\textwidth,0.1\textheight)
        \begin{beamercolorbox}[wd=6.3cm,ht=7.7cm,sep=0.5cm]{hvidbox}
            \fontsize{4}{10}\fontfamily{ptm}\selectfont \textls[200]{University of Copenhagen}
            \noindent\textcolor{KUrod}{\rule{5.3cm}{0.4pt}}
        \end{beamercolorbox}
    \end{textblock*}
    \begin{textblock*}{\textwidth}(0\textwidth,0.1\textheight)
        \begin{beamercolorbox}[wd=6.3cm,sep=0.5cm]{hvidbox}
                \textcolor{KUrod}{Learning Local Gibbs States from Noisy Shallow Quantum Circuits}
                \vspace{0.5cm}
                \par
                Department of Mathematical Sciences
                \vspace{0.5cm}
                \par
                \normalsize Daniel Gonzálvez Alfert  21/01/2026
        \end{beamercolorbox}
    \end{textblock*}
    \begin{textblock}{1}(6,11.44)
        \includegraphics[width=1cm]{KU/KU-logo.png}
    \end{textblock}
\end{frame}
}
\begin{frame}[hoved]
\frametitle{Outline}
\tableofcontents

\end{frame}

\section{Context and motivation}


\begin{frame}[hoved]
\frametitle{Noisy quantum circuits}

\begin{itemize}
    \item Due to the nature of quantum states, verifying the output of quantum devices is an arduous process.
    \item Quantum State Tomography (QST) aims to rebuild a quantum state through measurements.
    \item We need infinite measurements (and therefore copies) of a quantum state to know it exactly. In practice we only need certain precision.
    \item The cost of achieving the same precision increases exponentially with the number of qubits. 
\end{itemize}

\end{frame}



\begin{frame}[hoved]
\frametitle{Quantum State Tomography}

\begin{itemize}
    \item Due to the nature of quantum states, verifying the output of quantum devices is an arduous process.
    \item Quantum State Tomography (QST) aims to rebuild a quantum state through measurements.
    \item We need infinite measurements (and therefore copies) of a quantum state to know it exactly. In practice we only need certain precision.
    \item The cost of achieving the same precision increases exponentially with the number of qubits. 
\end{itemize}

\end{frame}
\input{sections/motivating_theorem}
%\section{Numerical results}
%\section{Heisenberg picture, $k$-designs}
%\section{Gibbs state truncation - conjecture}
%\section{Conclusion}


\end{document}